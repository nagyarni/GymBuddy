\chapter{Bevezetés}
\label{ch:bevezetes}

\section{Dolgozat felépítése}

\begin{description}
\item[I. Bevezetés:] Ebben a fejezetben részletesen kifejtem a szakdolgozat témáját, valamint bemutatom annak hátterét és motivációját.

\item[II. Felhasználói dokumentáció:] Ez a rész szolgál a program használatba vételéhez szükséges információkkal. Kiemelten foglalkozom a programkövetelményekkel, valamint az oldal funkcióinak részletes leírásával, melyeket képernyőképek és egy oldaltérkép segít megérteni.

\item[III. Fejlesztői dokumentáció:] Ez a rész a weboldal funkcióinak részletes elemzésére és leírására összpontosít. Usecase diagramok és User-story táblázatok segítségével világosan bemutatom a funkcionalitásokat. Továbbá részletesen tárgyalom a fejlesztés során alkalmazott keretrendszereket, tervezési mintákat, adatbázis felépítést és adattípusokat. A fejezet végén megtalálható a weboldal lokális futtatásához szükséges követelmények és lépéssorozat, valamint a tesztelési terv is.

\item[IV. Összefoglalás:] Ebben a záró fejezetben összegzem a szakdolgozatban megvalósított feladatot és annak eredményeit. Továbbá kitérek a lehetséges továbbfejlesztési lehetőségekre.

\item[V. Hivatkozások:] Az utolsó fejezetben felsorolom a szakdolgozat elkészítéséhez felhasznált programokat és dokumentációkat, valamint ezekre mutató hivatkozásokat.
\end{description}

\section{Témabejelentő}

A szakdolgozatom célja egy olyan telefonos/webes applikáció készítése, amelynek fő funkciója különböző edzéstervek készítése, valamint követése.

Lehetőség van klienseknek edzőkkel együttműködni egy belépési rendszer segítségével.
Az alkalmazás három fajta felhasználói fiókkal rendelkezik: Kliens, Edzői és Adminisztrációs.
A kliensek rögzíthetnek edzés során bizonyos adatokat az alkalmazásban, így lehetővé téve az edzéseik egyszerű követését.

A felhasználóknak rendelkezésre áll továbbá egy chat funkció is, amely használatával a kliensek képesek kommunikálni edzőkkel az alkalmazáson belül.
Edzéstervek könnyedén összeállíthatóak, akár az eddigi rögzített edzéstervek alapján automatikus válogatással.
Az adatok tárolása mögött egy központi adatbázis áll, így a felhasználó el tudja érni terveit, függetlenül a használt eszköztől.

Az alkalmazás e-mail értesítő rendszerrel is rendelkezik.
Lehetőség van az edzésterv pdf formátumban való exportálására, így akár ki is nyomtathatjuk azt.


\section{Motiváció}

Egyetemi éveim elején kezdtem erőemeléssel fogalkozni hobbiként, és igazán megragadott a sportág. Edzéseim mindig előre meg vannak tervezve edzőm közreműködésével, így segítve a gyors fejlődést, tapasztalt vagyok meglévő és gyakran használt eszközökkel amelyek edzés tervezési célra vannak használva világszerte. Az egyik ilyen elterjedt szoftver például egy egyszerű Excel-szerű táblázat, hiszen itt minden számadatot könnyedén rögzíteni tudunk, illetve különféle képletek alkalmazásával könnyíthetjük a munkánkat. Viszont számomra az ilyen szoftverek sok olyan funkcióval rendelkeznek, amelyeket használatuk során nem fogunk hasznosítani.

A probléma megoldása érdekében saját alkalmazást fejlesztettem, melyet úgy terveztem, hogy egyszerűsítse az edzők és kliensek munkáját az erőemelés terén. Az alkalmazás megkíméli őket a felesleges funkcióktól, ezzel optimalizálva a felhasználói élményt és hatékonyságot.