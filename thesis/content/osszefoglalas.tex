\chapter{Összefoglalás}
\label{ch:osszefoglalas}

Programom írása során az összes tervezett funkció megvalósításra került, összességében elégedett vagyok a végeredménnyel. A feladat komplexitása a számítottnál nagyobbnak bizonyult, ugyanis a szoftver minden részében volt olyan dolog, amivel még nem találkoztam. Egyetemi tanulmányaim során alap szinten megismerkedtem a React és a Node.js keretrendszerekkel, azonban a szakdolgozat írása közben mélyebb betekintést nyertem ezekbe a technológiákba. Számos új és eddig ismeretlen függvénykönyvtárat is alkalmaztam, például a Material UI-t. Ennek köszönhetően a jövőbeli webfejlesztési projekteim során magabiztosan fogok mozogni a React és Node.js környezetében.


% Szakdolgozatom írása során viszont nagyon sok új tapasztalatot szereztem ezekkel a technológiákkal kapcsolatban. Több új, számomra eddig ismeretlen függvénykönyvtárat is igénybe vettem, mint például a Material UI. Ennek köszönhetően jövőbeli webbel kapcsolatos projekteim feljesztése során sokkal otthonosabban fogom magamat érezni a React és NodeJs világában.

\bigskip

Az architektúrát szerintem sikerült jól rétegelni különböző részekre, amely a választott keretrendszereknek részében köszönhető (a szerver és kliens oldal két, teljesen különálló szerveren fut), viszont ezen felül a könyvtár struktúrát és az üzleti logika szétválasztását is igyekeztem minél áltáthatóbban rendezni és elégedett vagyok az eredménnyel.

\bigskip

Következő webes projektemnél kevesebb hangsúlyt fektetnék a kliensoldal fejlesztésére az elején, mert nagyobb újraformázásokra volt szükség amikor a kliens- és szerveroldalt összekötöttem egymással. Ezért érdemesebb lett volna a két progamrészen párhuzamosan dolgozni.

\pagebreak

\section{Továbbfejlesztési lehetőségek}

Programomat úgy építettem fel, hogy könnyedén bővíthető legyen a felhasználói interfész. Új funkciókat, például egy táplálkozás követő funkciót, vagy edzéssel kapcsolatos statisztikákat hozzáadni nem járna refaktorálásokkal.

\bigskip

Ezek mellett esetleg egy natív telefonos applikáció készítése a jobb felhasználói élményért is egy továbbfejlesztési lehetőség lenne. A webes alkamazásom egy előnye, hogy számítógépen és mobil eszközökön is egyaránt működik.

